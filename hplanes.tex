\documentclass{article}
\usepackage[letterpaper]{geometry}
%\usepackage[spacing,kerning]{microtype}
\usepackage{amsmath}
\usepackage{amsfonts}
\usepackage[pdftex,colorlinks,bookmarks=false]{hyperref}
%\usepackage{url}

\usepackage[pdftex]{color,graphicx}
\newcommand{\todo}[1]{\colorbox{red}{\begin{minipage}{\textwidth}{#1}\end{minipage}}}

\title{\todo{FANCY TITLE}}
\author{BarbaraJoy Jones}
\date{4 December 2009}

\begin{document}
\maketitle

\section{Intro?}
blah blah blah but we're living in a Euclidean world.\footnote{And I am a Euclidean girl.\cite{madonna}} blah blah shape of the universe.

\section{Curvature}
In the $\mathbb{R}^2$ plane, straight lines have zero curvature and circles have constant curvature.
For a circle of radius $R$, the curvature can be defined as $\frac{1}{R}$.
This is fairly intuitive: small circles curve a lot, while arcs of very large circles appear to be nearly straight.\cite{adventures}
For smooth curves without constant curvature, the curvature must be defined at a point $P$ rather than for the curve as a whole.
If we choose points on either side of $P$, we can find the circle that passes through the three points.
Choosing points closer and closer to $P$ will produce better approximations of the curve.
The curvature at $P$ is defined by $\kappa(P) = \frac{1}{R}$, where $R$ is the radius of the osculating circle (the circle that is the best approximation of the curve).
The sign of $\kappa(P)$ can be determined by choosing the direction of the normal vector at $P$.
By convention, the curvature is positive if the curve lies on one side of the line tangent to $P$ and the normal vector points to the same side.
If the normal vector points to the opposite side of the tangent line as the curve, the curvature is negative.
The curve will cross the tangent line at $P$ if $\kappa(P)$ is zero.\cite{singer}

A similar approach can be used for the curvature of surfaces in $\mathbb{R}^3$, as shown by Euler.
Given a smooth surface $\Sigma$ and a point $P$ on $\Sigma$, we define the normal line $l$ as the line through $P$ perpendicular to the tangent plane at $P$.
We then form the intersection of $\Sigma$ with a plane that contains $l$ and compute the curvature $\kappa$ of the intersection curve (i.e., the normal section) in that plane, given by
\[ \kappa = \kappa_1 \cos^2 \theta + \kappa_2 \sin^2 \theta \]
where $\kappa_1$ is the largest curvature, $\kappa_2$ is the smallest curvature, and $\theta$ is the angle between the planes corresponding to $\kappa_1$ and $\kappa$.
The planes corresponding to $\kappa_1$ and $\kappa_2$ are orthogonal.
The sign of the normal curvature is dependent upon the choice of the normal.
Replacing the normal line $l$ with a normal vector to the surface allows us to fix the sign of the curvature.\cite{singer}
This is a measure of \emph{extrinsic} curvature, because Euler's method depends on knowing how the surface is embedded in space.\cite{adventures}
\subsection{Gaussian (intrinsic) curvature}
\subsection{3D curvature}

\section{Models of the hyperbolic plane}
\subsection{Poincar\'e disc model}
\subsection{Klein model (projective disc model)}
\subsection{Poincar\'e half-plane model}
\subsection{Lorentz model (hyperboloid model)}

\section{Constructing the hyperbolic plane}
Here we shall define a construction as an approximation of an isometric embedding of the hyperbolic plane as a surface in $\mathbb{R}^3$ \cite{crochetplane}, as opposed to compass-and-straightedge constructions.
``Isometric constructions of the hyperbolic plane (or approximations of the hyperbolic plane) as surfaces in 3-space.'' \cite{crochetplane}

Embedding a hyperbolic plane in 3-space:\\
Given a hyperbolic plane $\mathcal{H}$ and the Euclidean space $\mathbb{R}^3$, there exists a Gauss map $\gamma$ such that
\[ \gamma : \mathcal{H} \hookrightarrow \mathbb{R}^3 \]
is an embedding (an injective and structure-preserving map).

\subsection{The annular hyperbolic plane}
The standard paper-and-tape construction using identical annular strips.
\subsection{basic crochet version}
best ratio is 12:13 (increase in every 12th stitch) http://www.math.cornell.edu/~dtaimina/hypplanes.htm
\subsection{polyhedral annular hyperbolic plane (paper)}
\subsection{hyperbolic soccer ball (paper)}
\subsection{defining the radius (curvature) of hyperbolic planes}
\subsection{crochet a pseudosphere}

\section{Proof that the constructions produce a hyperbolic plane}
The construction satisfies these major descriptions of hyperbolic planes
\subsection{Euclid's Postulates}
The first four postulates hold, while the fifth postulate does not.
\subsection{Pseudosphere}
Same intrinsic geometry as the pseudosphere
\subsection{Riemannian manifold with constant negative Gaussian curvature}
\subsection{upper half-plane model}

\section{topology!}
this should/could be really fucking cool\\
but can i include it? do i know enough?

\section{hyperbolic geometry in higher dimensions?}
All previous stuff is in $\mathbb{R}^2$? Maybe $\mathbb{R}^3$?

\section{historical/philosophical content: other combinations of fiber arts and non-euclidean geometry}
\subsection{Crocheting the Lorenz Manifold \cite{crochetlorenz}}
\subsection{Knitting non-Euclidean Pants \cite{makingmath}}
\subsection{Scarves and other M\"obius Knitting \cite{magicalknitting}\cite{magicalknitting2}}

\section{appendix: basic crochet info}
Instructions on standard crochet stitches, particularly those used for constructing hyperbolic planes. \cite{happyhooker}

\newpage
\bibliography{math}
\bibliographystyle{amsplain}

\end{document}
