\documentclass{article}
\usepackage[letterpaper]{geometry}
%\usepackage[spacing,kerning]{microtype}
\usepackage{amsmath}
\usepackage{amsfonts}
\usepackage[pdftex,colorlinks,bookmarks=false]{hyperref}
%\usepackage{url}

\title{Math 361 Rough Draft}
\author{BarbaraJoy Jones}
\date{16 October 2009}

\begin{document}
\maketitle

\section{Hyperbolic geometry as explained to my mother}
%\subsection{Euclid's Postulates}
%\subsection{Statements equivalent to the fifth postulate}
\subsection{Curvature}
\subsection{Gaussian (intrinsic) curvature}
\subsection{2D curvature}
\subsection{3D curvature}

\section{Models of the hyperbolic plane}
\subsection{Poincar\'e disc model}
\subsection{Klein model (projective disc model)}
\subsection{Poincar\'e half-plane model}
\subsection{Lorentz model (hyperboloid model)}

\section{Constructing the hyperbolic plane}
``Isometric constructions of the hyperbolic plane (or approximations of the hyperbolic plane) as surfaces in 3-space.'' \cite{crochetplane}
\subsection{The annular hyperbolic plane}
The standard paper-and-tape construction using identical annular strips.
\subsection{basic crochet version}
best ratio is 12:13 (increase in every 12th stitch) http://www.math.cornell.edu/~dtaimina/hypplanes.htm
\subsection{polyhedral annular hyperbolic plane (paper)}
\subsection{hyperbolic soccer ball (paper)}
\subsection{defining the radius (curvature) of hyperbolic planes}
\subsection{crochet a pseudosphere}

\section{Proof that the constructions produce a hyperbolic plane}
The construction satisfies these major descriptions of hyperbolic planes
\subsection{Euclid's Postulates}
The first four postulates hold, while the fifth postulate does not.
\subsection{Pseudosphere}
Same intrinsic geometry as the pseudosphere
\subsection{Riemannian manifold with constant negative Gaussian curvature}
\subsection{upper half-plane model}

\section{topology!}
this should/could be really fucking cool\\
but can i include it? do i know enough?

\section{hyperbolic geometry in higher dimensions?}
All previous stuff is in $\mathbb{R}^2$? Maybe $\mathbb{R}^3$?

\section{historical/philosophical content: other combinations of fiber arts and non-euclidean geometry}
\subsection{Crocheting the Lorenz Manifold \cite{crochetlorenz}}
\subsection{Knitting non-Euclidean Pants \cite{makingmath}}
\subsection{Scarves and other M\"obius Knitting \cite{magicalknitting}\cite{magicalknitting2}}

\section{appendix: basic crochet info}
Instructions on standard crochet stitches, particularly those used for constructing hyperbolic planes. \cite{happyhooker}

\newpage
\bibliography{math}
\bibliographystyle{amsplain}

\end{document}
