\documentclass[letterpaper,titlepage]{article}
\usepackage[letterpaper]{geometry}
%\usepackage[spacing,kerning]{microtype}
\usepackage{amsmath}
\usepackage{amsfonts}
\usepackage[pdftex,colorlinks,bookmarks=false]{hyperref}
%\usepackage{url}

\usepackage[pdftex]{color,graphicx}
\newcommand{\todo}[1]{\colorbox{red}{\begin{minipage}{\textwidth}{#1}\end{minipage}}}
\newcommand{\esp}{$\mathbb{R}^3$}

\title{Combining non-Euclidean Geometry and Fiber Arts:\\
\bigskip
\Large{Crocheting constructions of the hyperbolic plane in \esp}}
\author{BarbaraJoy Jones}
\date{4 December 2009}

\begin{document}
\maketitle


\section{Curvature}
\subsection{Extrinsic curvature}
In the $\mathbb{R}^2$ plane, straight lines have zero curvature and circles have constant curvature.
For a circle of radius $R$, the curvature can be defined as $\frac{1}{R}$.
This is fairly intuitive: small circles curve a lot, while arcs of very large circles appear to be nearly straight.\cite{adventures}
For smooth curves without constant curvature, the curvature must be defined at a point $P$ rather than for the curve as a whole.
If we choose points on either side of $P$, we can find the circle that passes through the three points.
Choosing points closer and closer to $P$ will produce better approximations of the curve.
The curvature at $P$ is defined by
\begin{equation}
\kappa(P) = \frac{1}{R}
\end{equation}
where $R$ is the radius of the osculating circle (the circle that is the best approximation of the curve).
The sign of $\kappa(P)$ can be determined by choosing the direction of the normal vector at $P$.
By convention, the curvature is positive if the curve lies on one side of the line tangent to $P$ and the normal vector points to the same side.
If the normal vector points to the opposite side of the tangent line as the curve, the curvature is negative.
The curve will cross the tangent line at $P$ if $\kappa(P)$ is zero.\cite{singer}
This means of determining curvature requires us to observe how a one-dimensional curve is embedded in the $\mathbb{R}^2$ plane, so it is an \emph{extrinsic} measure.

A similar approach can be used for the curvature of surfaces in \esp, as shown by Euler.
Given a smooth surface $\Sigma$ and a point $P$ on $\Sigma$, we define the normal line $l$ as the line through $P$ perpendicular to the tangent plane at $P$.
We then form the intersection of $\Sigma$ with a plane that contains $l$ and compute the curvature $\kappa$ of the intersection curve (i.e., the normal section) in that plane, given by
\begin{equation}
\kappa = \kappa_1 \cos^2 \theta + \kappa_2 \sin^2 \theta
\label{euler}
\end{equation}
where $\kappa_1$ is the largest curvature, $\kappa_2$ is the smallest curvature, and $\theta$ is the angle between the planes corresponding to $\kappa_1$ and $\kappa$.
The planes corresponding to $\kappa_1$ and $\kappa_2$ are orthogonal.
The sign of the normal curvature is dependent upon the choice of the normal.
Replacing the normal line $l$ with a normal vector to the surface allows us to fix the sign of the curvature.\cite{singer}
This is also a measure of extrinsic curvature, as Euler's method depends on knowing how the surface is embedded in space.\cite{adventures}

\subsection{Gaussian curvature}
Consider an ant on a sphere.
As she crawls around the surface of the sphere, the ant can only observe the surface in two dimensions.
The ant can still determine that she is living on a sphere without measuring extrinsic curvature.
If she ties a short rope of length $r$ to a post at the point $P$, holds the other end of the rope, and walks in a circle around the post with the rope held taut, she can measure the distance $\delta$ around the circle.
She knows that the circumference of a Euclidean circle, so she can determine that she lives on a sphere if $\delta < 2\pi r$.\cite{adventures}
The ant observes the intrinsic properties of the surface to come to this conclusion, rather than external knowledge of the shape.
The amount that the circumference of a circle differs from $2\pi r$ is called the angle defect.\footnote{Or the angle excess, if greater than $2\pi r$.}
This allows us to define the intrinsic curvature at a point $P$ to be
\begin{equation}
K_P = \lim_{r \to 0} \frac{\theta_r}{A_r}
\label{circle}
\end{equation}
where $\theta_r$ is the angle defect of a circle centered at $P$ and $A_r$ is the area of the circle.
If $K_P \not= 0$ (i.e., the surface is not flat), then $A_r \not= \pi r^2$.\cite{makingmath}

Gauss defined the intrinsic curvature of a surface at a point $P$ as
\begin{equation}
K = \kappa_1\kappa_2
\label{gauss}
\end{equation}
where $\kappa_1$ and $\kappa_2$ are the principal curvatures\footnote{The principal curvatures $\kappa_1,\kappa_2$ are the same as the curvatures of the intersection curves in Euler's definition of extrinsic curvature.} at $P$.\cite{gauss}
While \eqref{circle} is equivalent to \eqref{gauss}, the Gaussian curvature as defined in \eqref{gauss} is rather more useful.
The sign of $K$ is not affected by how we choose the direction of the normal vector when finding $\kappa_1,\kappa_2$, as long as we are consistent; both will change signs if we reverse the normal vector.

Spheres, like circles, have constant curvature.
$K$, the product of the principal curvatures, will always be positive since $\kappa_1 = \kappa_2$.
A point $P$ on a sphere $\Sigma$ with radius $r$ is determined by the intersection of two great circles, also with radius $r$.
The curvature of each of the great circles is $\frac{1}{r}$ as defined above, so the curvature at $P$ and other points on $\Sigma$ is $\frac{1}{r^2}$.
An intuitive interpretation of ``positive curvature'' is that all of the normal sections curve in the same direction, meaning the surface is convex.\cite{singer}

A hyperbolic plane $\mathcal{H}$ will also have constant curvature.
Unlike spheres, however, one of the principal curvatures will be positive while the other is negative.
By \eqref{gauss}, $\mathcal{H}$ has \emph{negative} constant curvature
\begin{equation}
K = \frac{-1}{\rho^2}
\label{hcurve}
\end{equation}
where $\rho$ is the radius of the hyperbolic plane. This will be further discussed in \ref{hradius}.

The \emph{Theorema Egregium} follows from the definition of curvature.
It states that the Gaussian curvature is an intrinsic invariant of the surface.\cite{gauss}
The manner in which the surface is embedded in $\mathbb{R}^3$ does not affect the determination of curvature.
Extending this to the Gauss-Bonnet Theorem means that a change in the embedding of a surface that changes the curvature at some point $P_1$ will also change the curvature at some other point $P_2$ such that there is no net change in the curvature of the surface.\cite{makingmath}

\section{Models of the hyperbolic plane}
\subsection{Poincar\'e disc model}
\subsection{Klein model (projective disc model)}
\subsection{Poincar\'e half-plane model}
\subsection{Lorentz model (hyperboloid model)}

\section{Constructing the hyperbolic plane}
Why construct hyperbolic planes in \esp?
Discussions on the shape of the universe aside, we are living in a Euclidean world.\footnote{And I am a Euclidean girl.\cite{madonna}}
Representing the various models described above can be very difficult to do in an intuitive manner.
Just as attempting to map the surface of a sphere to a plane leads to the sort of distortion that causes Greenland to appear as large as Africa, mapping the hyperbolic plane to $\mathbb{R}^2$ is also distorted.\cite{makingmath}
This distortion has been used for artistic effect by M. C. Escher and Carlo S\'equin, among others, but it tends to make explorations of hyperbolic geometry more difficult.\cite{adventures}

As a consequence of the \emph{Theorema Egregium}, it is impossible to isometrically embed the hyperbolic plane as a complete subset of \esp.\cite{gauss}
\begin{quote}
Hilbert proved that there is no real analytic isometric
embedding of the hyperbolic plane onto a complete
subset of 3-space, and his arguments also work to show
that there is no isometric embedding whose derivatives up
to order four are continuous. Moreover, in 1964, N. V.
Efimov extended Hilbert's result by proving that there is no
isometric embedding defined by functions whose first and
second derivatives are continuous. However, in 1955, N.
Kuiper proved that there is an isometric embedding with
continuous first derivatives of the hyperbolic plane onto a
closed subset of 3-space.\cite{crochetplane}
\end{quote}
Henderson and Taimina began with Thurston's work\cite{thurston} to describe finite surfaces that can apparently be extended indefinitely, but and not always differentiably embedded.
 
Given a hyperbolic plane $\mathcal{H}$ and the Euclidean space \esp, there exists a Gauss map $\gamma$ such that
\begin{equation}
\gamma : \mathcal{H} \hookrightarrow \mathbb{R}^3
\label{gaussmap}
\end{equation}
is an embedding (an injective and structure-preserving map).
We shall define a construction as an approximation of an isometric embedding \eqref{gaussmap} of the hyperbolic plane as a surface in \esp \cite{crochetplane}, as opposed to compass-and-straightedge constructions.\footnote{These constructions could also be described as yarn-and-crochet-hook constructions.}

\subsection{The annular hyperbolic plane}
William Thurston first developed a construction of the hyperbolic plane by using paper annuli held together with tape.\cite{thurston}
An annulus is the region between two concentric circles; an annular strip is a portion of an annulus cut off by an angle from the center of the circles.\cite{crochetplane}
An approximation of the hyperbolic plane is constructed by cutting out of paper several identical annular strips of radius $\rho$ and thickness $\delta$.\footnote{David Henderson's template can be downloaded here: http://www.math.cornell.edu/$\sim$henderson/ExpGeom/annuli.jpg}
The strips are attached together with the inner circle of one taped to the outer circle of another or with the straight ends together.
Holding $\rho$ fixed while letting $\delta \to 0$ allows us to obtain the actual annular hyperbolic plane.\cite{crochetplane}\cite{adventures}
\subsection{Crocheting the hyperbolic plane}
Choose a yarn that is not very stretchy (cheap acrylic yarn works very well) and a crochet hook slightly smaller than the size recommended on the label.
\begin{itemize}
\item Work 10-20 chain stitches.
\item For each row, single crochet for the first $n$ stitches.
\item The $(n+1)^{\text{st}}$ stitch should be worked as an increase (single crochet into the same loop as the $n^{\text{th}}$ stitch).
\item Continue in this manner to the end of each row, work a chain stitch, and begin the next row.
\item Repeat until the model is as large as desired.
\end{itemize}
This method is very easy to use, but it is definitely an approximation.
It cannot be described by analytic equations.
A more precise approximation can be produced by examining the geometry of the pseudosphere and performing some calculations based on the size of a single crochet stitch.

\subsection{The pseudosphere and symmetric hyperbolic planes}
A pseudosphere is the surface generated by rotating a tractrix about its asymptote.
In 1868, Eugenio Beltrami proved that hyperbolic geometry holds locally on the pseudosphere.
To crochet the pseudosphere, we would ideally start from a point.
This is not physically possible, so instead we start with only a few chain stitches to make a tiny circle.
We then crochet in a spiral, increasing with some constant ratio as described above.
The project will initially form a cone shape, then it will flare out and form ruffles.
The cone will be absorbed by the ruffles if the pseudosphere is made large enough.

A symmetric hyperbolic plane is produced by using the pseudosphere method and adjusting the rate of increase to be sure the surface will have constant negative curvature. We begin with the equation
\begin{equation}
C = 2 \pi \rho \sinh \left( \frac{r}{\rho} \right)
\label{Csinh}
\end{equation}
which can also be written as
\begin{equation}
C = \pi \rho \left( e^{r/\rho} - e^{-r/\rho} \right)
\label{Ce}
\end{equation}
where $\rho$ is the radius of the hyperbolic plane to be crocheted, $r$ is the intrinsic radius of a circle (intrinsic meaning measured along the surface of the hyperbolic plane; a symmetric hyperbolic plane will consist of crocheting ``concentric'' intrinsic circles), and $C$ is the intrinsic circumference of a circle with intrinsic radius $r$ on a hyperbolic plane with radius $\rho$.
Since $r$ depends on the height of a crocheted row $h$, the intrinsic radius of the $n$th row is $r_n = nh$.
For each row, the intrinsic circumference $C(n)$ is
\begin{equation}
C(n) = \pi \rho \left( e^{nh/\rho} - e^{-nh/\rho} \right)
\label{Cn}
\end{equation}
The ratio $\frac{C(n)}{C(n-1)}$ determines how to increase stitches.
This needs to be a fraction of the form $\frac{(k+1)}{k}$, where $k$ is an integer, to crochet the plane.
The number of stitches in the $n^{\text{th}}$ row is determined by $S(n) = \frac{C(n)}{w}$, where $w$ is the width of one stitch.
We can use \eqref{Cn} and the ratio $\frac{C(n)}{C(n-1)}$ to construct a table determining the increase ratio for each row.
As the plane grows, the increase ratio will eventually stabilize.
While $r$ is small, the $e^{-r/\rho}$ term in \eqref{Cn} is significant.
When $r$ is large enough, we find the limit $\lim_{r \to \infty} = 0$ and \eqref{Cn} becomes
\begin{equation}
C(n) = \pi \rho e^{nh/\rho}
\end{equation}
for sufficiently large $n$.

\subsection{Radius and curvature of hyperbolic planes}
\label{hradius}
The radius of an annular hyperbolic plane is defined as $\rho$, the radius of the annuli.
We obtain different hyperbolic planes depending on the value of $\rho$; small values produce very ``ruffly'' planes while larger values produce less ruffly planes.
As $\rho$ increases, the plane has less curvature and hence becomes flatter.
In fact, as $\rho$ goes to infinity the plane becomes indistinguishable from the Euclidean plane.
This also holds true for the sphere.

\section{Proof that the constructions produce a hyperbolic plane}
The construction satisfies these major descriptions of hyperbolic planes
\subsection{Euclid's Postulates}
The first four postulates hold, while the fifth postulate does not.
\subsection{Pseudosphere}
Same intrinsic geometry as the pseudosphere
\subsection{Riemannian manifold with constant negative Gaussian curvature}
\subsection{upper half-plane model}

\section{historical/philosophical content: other combinations of fiber arts and non-euclidean geometry}
%\subsection{Other things that are hyperbolic planes\cite{adventures}}
\subsection{Crocheting the Lorenz Manifold \cite{crochetlorenz}}
\subsection{Knitting non-Euclidean Pants \cite{makingmath}}
%\subsection{Scarves and other M\"obius Knitting \cite{magicalknitting}\cite{magicalknitting2}}

\section*{appendix: basic crochet info}
Instructions on standard crochet stitches, particularly those used for constructing hyperbolic planes. \cite{happyhooker}

\newpage
\bibliography{math}
\bibliographystyle{amsplain}

\end{document}
