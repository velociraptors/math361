\documentclass{article}
\usepackage[letterpaper]{geometry}
\usepackage{amsmath}
\usepackage{amsfonts}
\usepackage[pdftex,colorlinks,bookmarks=false]{hyperref}

\author{BarbaraJoy Jones}

\begin{document}

\section*{Materials}
\subsection*{Yarn}
Cascade 220
\subsection*{Notions}
Crochet hook
\subsection*{Gauge}

\section*{Abbreviations}

\section*{Pattern}

\[ C = 2 \pi R \sinh (r/R) \]

\[ C = \pi R (e^{r/R} - e^{-r/R}) \]

\begin{itemize}
\item $R$ is the radius of the hyperbolic plane to be crocheted (radius of the annuli)
\item $r$ is the intrinsic radius of a circle (intrinsic meaning measured along the surface of the hyperbolic plane; a symmetric hyperbolic plane will consist of crocheting ``concentric'' intrinsic circles)
\item $C$ is the intrinsic circumference of a circle with intrinsic radius $r$ on a hyperbolic plane with radius $R$
\item sinh is the hyperbolic sine function
\end{itemize}
Since $r$ depends on the height of a crocheted row $h$, the intrinsic radius of the $n$th row is $r_n = nh$. For each row, the intrinsic circumference $C(n)$ is
\[ C(n) = \pi R (e^{nh/R} - e^{-nh/R})\]
The ratio $C(n)/C(n-1)$ determines how to increase stitches. This needs to be a fraction of the form $(k+1)/k$ to crochet the plane. The number of stitches in the $n$th row is determined by $S(n) = C(n)/w$, where $w$ is the width of one stitch.

Constants:\\
$R$ = 8.0 cm\\
$h$ = SOMETHING\\
$w$ = SOMETHING\\

\begin{tabular}{|c|c|c|c|c|c|c|}
\hline $n$ & $C(n)$ & $C(n)/C(n-1)$ & Nearby fractions & Increase ratio & $S(n)$ & increases \\ 
\hline 1 &  &  &  &  &  &  \\ 
\hline 2 &  &  &  &  &  &  \\ 
\hline 3 &  &  &  &  &  &  \\ 
\hline 4 &  &  &  &  &  &  \\ 
\hline 5 &  &  &  &  &  &  \\ 
\hline 6 &  &  &  &  &  &  \\ 
\hline 7 &  &  &  &  &  &  \\ 
\hline 8 &  &  &  &  &  &  \\ 
\hline 9 &  &  &  &  &  &  \\ 
\hline 10 &  &  &  &  &  &  \\ 
\hline 11 &  &  &  &  &  &  \\ 
\hline 12 &  &  &  &  &  &  \\ 
\hline 13 &  &  &  &  &  &  \\ 
\hline 14 &  &  &  &  &  &  \\ 
\hline 15 &  &  &  &  &  &  \\ 
\hline 16 &  &  &  &  &  &  \\ 
\hline 17 &  &  &  &  &  &  \\ 
\hline 18 &  &  &  &  &  &  \\ 
\hline 19 &  &  &  &  &  &  \\ 
\hline 
\end{tabular} 

\end{document}